\chapter{Name}
\label{chapter:<name>}
\graphicspath{ {./chapterXX/Fig} }

\begin{itquote}
Stuff to quote
\end{itquote}

\section*{Learning Objectives}
\noindent\rule{\linewidth}{1pt}
By the end of this chapter, the reader should:


\begin{figure}
\includegraphics<replace with synthesized text>
\caption{}
\label{figure:}
\end{figure}


\begin{table}
\caption{}
\label{table:<context>}
\begin{tabular}{|l|m{10cm}|} \hline
row \\ \hline
row \\ \hline
row \\ \hline
\end{tabular}
\end{table}


\begin{equation}
\label{equ:integralDeltaSliceOfPDF}
R(t) = 1 - F(t)
\end{equation}


\begin{example}{Combination system reliability.}
\label{example:<contex>}
\emph{\textbf{\ul{Problem:}}} Consider.... \\	% Force line break for spacing
\noindent\emph{\textbf{\ul{Solution:}}} The parallel systems...
\end{example}


%%------------------------------------------------------
%%				Lessons Learned
%% If you want to use multirow with rowcolor:
%%		Color rows as usual
%%		Use multirow on the last row with negative row values
%%		You will need to use hhline for non-multirows
%%
%% Use 			\hphantom  \\		to insert a space between things
%%------------------------------------------------------
