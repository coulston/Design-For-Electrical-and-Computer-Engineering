\section{Problems}

\begin{enumerate}
\def\labelenumi{\arabic{enumi}.}
\item
  Describe the relationship between ethics and morals.
\item
  Describe the differences between morals and values.
\item
  Which patent is most relevant for engineering inventions, a design
  patent or utility patent? Why?
\item
  What are the criteria that are used in evaluating patents?
\item
  Explain the importance of claims in a patent application.
\item
  Discuss the tradeoffs involved between using patents and trade secrets
  to protect intellectual property.
\item
  When can reverse-engineering be used, and how can the information
  obtained from it be used?
\item
  What is the difference between negligence and strict liability in tort
  law?
\item
  For the case study presented below, apply the ethical decision making
  paradigm presented in Section~\ref{section:case-study-analysis} to 
analyze the situation. Present
  potential solutions to the scenario and provide a discussion of them.

\begin{quote}
\ul{\hfill\break
Case Study: Disk Drive Diagnostics. (Copyright John Wallberg. Reprinted
by permission.)}

SCSI, an industry standard system for connecting devices (like disks) to
computers, provides a vendor ID protocol by which the computer can
identify the supplier (and model) of every attached disk.

Company C makes file servers consisting of a processor and disks. Disks
sold by C identify C in their vendor ID. Disks from other manufacturers
can be connected to C\textquotesingle s file servers; however, the file
server software performs certain maintenance functions, notably
pre-failure warnings based on performance monitoring, only on C-supplied
disks.

Company P decides to compete with C by supplying cheaper disks for
C\textquotesingle s file server. They quickly discover that while their
disks work on C\textquotesingle s file servers, their disks lack a
pre-failure warning feature that C\textquotesingle s disks have.
Therefore, the CEO of P directs you, the engineer in charge of the disk
product, to find a solution to the problem of no pre-failure warning for
your disks. Using reverse engineering, you discover that by changing the
vendor ID of P's disks, the C file servers will treat P disks as C
disks. Your management at company P instructs you to incorporate this
change into your product so that you can advertise the disks as ``100\%
C-compatible.'' What would you do in this situation?
\end{quote}


\item
  For the case study presented below, apply the ethical decision making
  paradigm presented in Section~\ref{section:case-study-analysis} 
	to analyze the situation. Present
  potential courses of action and provide a discussion of them.

\begin{quote}
\ul{\hfill\break
Case Study: Encryption Software} (Texas A\&M Ethics Case Studies,
\url{http://ethics.tamu.edu}. Reprinted by permission.)\\

You are a recently hired engineer who has been recruited directly out of
college. For your first assignment, your boss asked you to write a piece
of software to provide security from "prying eyes" over e­mailed
documents; these documents would be used internally by the company. This
software will subsequently be distributed to different departments.

Upon completion of this software project, you saw a program on the local
news about an individual in California who has made similar software
available overseas. This individual is currently under prosecution in a
federal court for the distribution of algorithms and information which
(by law) must remain within the United States for purposes of national
security.

It occurs to you that your company is a multinational corporation and
that the software might have been distributed overseas. You then
discover that the software has indeed been sent overseas to other
offices within the corporation. You speak with your boss, informing him
of the news program from the night before. He shrugs off this comment,
stating that ``The company is based in the United States and we are
certainly no threat to national security in any way. Besides,
there\textquotesingle s no way anyone will find out about software we
use internally.''

You agree with your boss, and let it go. Later on however, you receive a
letter from a gentleman working as a contractor for his company
overseas. Through some correspondence regarding the functionality of the
software and technical matters, you learn that the Middle Eastern office
had been supplying his software outside the company to contractors and
clients so that they could exchange secure e­mailed documents. What would
you do in this situation?
\end{quote}


\item
  For the case study presented below, apply the ethical decision making
  paradigm presented in Section~\ref{section:case-study-analysis} 
to analyze the situation. Present
  potential courses of action and provide a discussion of them.

\begin{quote}
\ul{\hfill\break Case Study: A Failure.} (Texas A\&M Ethics Case Studies,
\url{http://ethics.tamu.edu}. Reprinted by permission.)

You work for Velky Measurement which has for years provided DGC
Corporation with sophisticated electronic equipment for patient health
monitoring systems. Recently, DGC returned a failed piece of measurement
equipment. A meeting was held with representatives of Velky and DGC to
discuss the problem. This included you and your project manager who is
intimately acquainted with the returned equipment. During the course of
the meeting it becomes apparent to you that the problem has to be
Velky\textquotesingle s. You suspect that the equipment failed because
of an internal design problem and that it was not properly tested.
However, at the conclusion of the meeting your project manager
represents Velky's official position---the test equipment is functioning
properly.

You keep silent during the meeting, but afterwards talk to your project
manager about his diagnosis. You suggest that Velky tell DGC that the
problem is due to a design fault and that Velky will replace the
defective equipment. You manager replies, ``\emph{I
don\textquotesingle t think it\textquotesingle s wise to acknowledge
that it\textquotesingle s our fault. There\textquotesingle s no need to
hang out our wash and lessen DGC\textquotesingle s confidence in the
quality of our work. A good will gesture to replace the equipment should
suffice}.''

Utlimately, Velky's management replaces the equipment because DGC has
been such a good customer. Although Velky replaces the equipment at its
own expense, it does not disclose the real nature of the problem. What
would you do in this situation?
\end{quote}

\item
  For the case study presented below, apply the ethical decision making
  paradigm presented in Section~\ref{section:case-study-analysis} to 
analyze the situation. Present
  potential courses of action and provide a discussion of them.

\begin{quote}
\ul{\hfill\break Case Study: A Vacation} (Texas A\&M Ethics Case Studies,
\url{http://ethics.tamu.edu}. Reprinted by permission.)

You work for Rancott and were looking forward to an upcoming trip for
weeks. Once you were assigned to help install Rancott's equipment for
Boulding Corporation, you arranged a vacation at a nearby ski resort.
The installation was scheduled to be completed on the
12\textsuperscript{th} and your vacation would begin on the
13\textsuperscript{th}. That meant a full week of skiing with three of
your old college buddies.

Unfortunately, not all of the equipment arrived on time. Eight of the
ten identical units were installed by mid-morning on the
12\textsuperscript{th}. Even if the remaining two units had arrived that
morning, it would take another full day to install them. However, you
were informed that it might take as long as two more days for the units
to arrive.

``\emph{Terrific,}'' you sighed, ``\emph{there goes my vacation---and
all the money I put down for the condo}.'' ``\emph{No problem},''
replied Jerry, the Boulding engineer who had worked side-by-side with
you as each of the first eight units was installed. He said ``\emph{I
can handle this for you. We did the first eight together.
It\textquotesingle s silly for you to have to hang around and blow your
vacation}.'' Jerry knew why you were sent to supervise the installation
of the new equipment. It had to be properly installed in order to avoid
risking injuries to those who use it. Although you are aware of this,
you are confident that Jerry is fully capable to supervise the
installation of the remaining two units. What would you do?
\end{quote}

\end{enumerate}