\section{Preface}\label{preface}

This book is written for undergraduate students and teachers engaged in
electrical and computer engineering (ECE) design projects, primarily in
the senior year. The objective of the text is to provide a treatment of
the design process in ECE with a sound academic basis that is integrated
with practical application. This combination is necessary in design
projects because students are expected to apply their theoretical
knowledge to bring useful systems to reality. This topical integration
is reflected in the subtitle of the book: Theory, Concepts, and
Practice. Fundamental theories are developed whenever possible, such as
in the chapters on functional design decomposition, system behavior, and
design for reliability. Many aspects of the design process are based
upon time-tested concepts that represent the generalization of
successful practices and experience. These concepts are embodied in
processes presented in the book, for example, in the chapters on needs
identification and requirements development. Regardless of the topic,
the goal is to apply the material to practical problems and design
projects. Overall, we believe that this text is unique in providing a
comprehensive design treatment for ECE, something that is sorely missing
in the field. We hope that it will fill an important need as capstone
design projects continue to grow in importance in engineering education.

We have found that there are three important pieces to completing a
successful design project. The first is an understanding of the design
process, the second is an understanding of how to apply technical design
tools, and the third is successful application of professional skills.
Design teams that effectively synthesize all three tend to be far more
successful than those that don't. The book is organized into three parts
that support each of these areas.

The first part of the book, the \emph{Design Process}, embodies the
steps required to take an idea from concept to successful design. At
first, many students consider the design process to be obvious. Yet it
is clear that failure to understand and follow a structured design
process often leads to problems in development, if not outright failure.
The design process is a theme that is woven throughout the text;
however, its main emphasis is placed in the first four chapters. Chapter
1 is an introduction to design processes in different ECE application
domains. Chapter 2 provides guidance on how to select projects and
assess the needs of the customer or user. Depending upon how the design
experience is structured, both students and faculty may be faced with
the task of selecting the project concept. Further, one of the important
issues in the engineering design is to understand that systems are
developed for use by an end-user, and if not designed to properly meet
that need, they will likely fail. Chapter 3 explains how to develop the
Requirements Specification along with methods for developing and
documenting the requirements. Practical examples are provided to
illustrate these methods and techniques. Chapter 4 presents concept
generation and evaluation. A hallmark of design is that there are many
potential solutions to the problem. Designers need to creatively explore
the space of possible solutions and apply judgment to select the best
one from the competing alternatives.

The second part of the book, \emph{Design Tools}, presents important
technical tools that ECE designers often draw upon. Chapter 5 emphasizes
system engineering concepts including the well known functional
decomposition design technique and applications in a number of ECE
problem domains. Chapter 6 provides methods for describing system
behavior, such as flowcharts, state diagrams, data flow diagrams and a
brief overview of the Unified Modeling Language (UML). Chapter 7 covers
important issues in testing and provides different viewpoints on testing
throughout the development cycle. Chapter 8 addresses reliability theory
in design, and reliability at both the component and system level is
considered.

The third part of the book focuses on \emph{Professional Skills}.
Designing, building, and testing a system is a process that challenges
the best teams, and requires good communication and project management
skills. Chapter 9 provides guidance for effective teamwork. It provides
an overview of pertinent research on teaming and distills it into a set
of heuristics. Chapter 10 presents traditional elements of project
planning, such as the work breakdown structure, network diagrams, and
critical path estimation. It also addresses how to estimate manpower
needs for a design project. Chapter 11 addresses ethical considerations
in both system design and professional practice. Case studies for ECE
scenarios are examined and analyzed using the IEEE (Institute of
Electrical and Electronics Engineers) Code of Ethics as a basis. The
book concludes with Chapter 12, which contains guidance for students
preparing for oral presentations, often a part of capstone design
projects.

\textbf{Features of the Book}

This book aims to guide students and faculty through the steps necessary
for the successful execution of design projects. Some of the features
are listed below.

\begin{itemize}
\item
  Each chapter provides a brief motivation for the material in the
  chapter followed by specific learning objectives.
\item
  There are many examples throughout the book that demonstrate the
  application of the material.
\item
  Each end-of-chapter problem has a different intention. Review problems
  demonstrate comprehension of the material in the chapter. Application
  problems require the solution of problems based upon the material
  learned in the chapter. Design problems are directly applicable to
  design projects and are usually tied in with the Project Application
  section.
\item
  Nearly all chapters contain a Project Application section that
  describes how to apply the material to a design project.
\item
  Some chapters contain a Guidance section that represents the author's
  advice on application of the material to a design project.
\item
  Checklists are provided for helping students assess their work.
\item
  There are many terms used in design whose meaning needs to be
  understood. The text contains a glossary with definitions of design
  terminology. The terms defined in the glossary (Appendix A) are
  indicated by \emph{\textbf{italicized-bold}} highlighting in the text.
\item
  All chapters conclude with a Summary and Further Reading section. The
  aim of the Further Reading portion is to provide pointers for those
  who want to delve deeper into the material presented.
\item
  The book is structured to help programs demonstrate that they are
  meeting the ABET (accreditation board for engineering programs)
  accreditation criteria. It provides examples of how to address
  constraints and standards that must be considered in design projects.
  Furthermore, many of the professional skills topics, such as teamwork,
  ethics, and oral presentation ability, are directly related to the
  ABET Educational Outcomes. The requirements development methods
  presented in Chapter 3 are valuable tools for helping students perform
  on cross-functional teams where they must communicate with
  non-engineers.
\item
  An instructor's manual is available that 
  contains not only solutions, but guidance from the authors on teaching
  the material and managing student design teams. It is particularly
  important to provide advice to instructors since teaching design has
  unique challenges that are different than teaching engineering science
  oriented courses that most faculty are familiar with.
\item
  PowerPoint\textsuperscript{TM} presentations are available for
  instructors through McGraw-Hill
\item
  There are a number of complete case study student projects available
  in electronic form for download by both students and instructors and
  available at.
  These projects have been developed using the processes provided in
  this book.
\end{itemize}

\textbf{How to Use this Book}

There are several common models for teaching capstone design, and this
book has the flexibility to serve different needs. Particularly,
chapters from the Professional Skills section can be inserted as
appropriate throughout the course. Recommended usage of the book for
three different models of teaching a capstone design course is
presented.

\begin{itemize}
\item
  \textbf{Model I.} This is a two-semester course sequence. In the first
  semester, students learn about design principles and start their
  capstone projects. This is the model that we follow. In the first
  semester the material in the book is covered in its entirety. The
  order of coverage is typically Chapters 1--3, 9, 4--6, 10--11, and
  7--8. Chapter 9 (Teams and Teamwork) is covered immediately after the
  projects are identified and the teams are formed. Chapters 10 (Project
  Management) and 11 (Ethical and Legal Issues) are covered after the
  system design techniques in Chapters 5 and 6 are presented. Students
  are in a good position to create a project plan and address ethical
  issues in their designs after learning the more technical aspects of
  design. Chapter 12 (Oral Presentations) is assigned to students to
  read before their first oral presentation to the faculty. The course
  concludes with principles of testing and system reliability (Chapter 7
  and 8). We assign a good number of end-of-chapter problems and have
  quizzes throughout the semester. By the end of the first semester,
  design teams are expected to have completed development of the
  requirements, the high-level or architectural design, and developed a
  project plan. In the second semester, student teams implement and test
  their designs under the guidance of a faculty advisor.
\item
  \textbf{Model II}. This two-semester course sequence is similar to
  Model I with the difference being that the first semester is a lower
  credit course (often one credit) taught in a seminar format. In this
  model chapters can be selected to support the projects. Some of the
  core chapters for consideration are Chapters 1--5, which take the
  student from project selection to functional design, and Chapters
  9--11 on teamwork, project management, and ethical issues. Other
  chapters could be covered at the instructor's discretion. The use of
  end-of-chapter problems would be limited, but the project application
  sections and example problems in the text would be useful in guiding
  students through their projects.
\item
  \textbf{Model III}. This is a one-semester design sequence. Here, the
  book would be used to guide students through the design process.
  Chapters for consideration are 1--5 and 9--10, which provide the
  basics of design, teamwork, and project management. The project
  application sections and problems could be used as guidance for the
  project teams.
\end{itemize}

\textbf{Acknowledgements}

Undertaking this work has been a challenging experience and could not
have been done without the support of many others. First, we thank our
families for their support and patience. They have endured many hours
and late evenings that we spent researching and writing. Melanie Ford is
to be thanked for her diligent proofreading efforts. Bob Simoneau, the
former Director of the Penn State Behrend School of Engineering, has
been a great supporter of the book and has also lent his time in reading
and providing comments. Our school has a strong design culture, and this
book would not have happened without that emphasis; our faculty
colleagues need to be recognized for developing that culture. Jana
Goodrich and Rob Weissbach are two faculty members with whom we have
collaborated on other courses and projects. They have influenced our
thinking in this book, particularly in regard to project selection,
requirements development, cost estimation, and teamwork. We must also
recognize the great collaborative working environment that exists at
Penn State Behrend, which has allowed this work to flourish. Our
students have been patient in allowing us to experiment with different
material in the class and on the projects. Examples of their work are
included in the book and are greatly appreciated. John Wallberg
contributed the disk drive diagnostics case study in Chapter 11 that we
have found very useful for in-class discussions. John developed this
while he was a student at MIT. Thanks to Anne Maloney for her
copyediting of the manuscript. The following individuals at McGraw-Hill
have been very supportive and we thank them for their efforts to make
this book a reality -- Carlise Stembridge, Julie Kehrwald, Darlene
Schueller, Craig Marty, Kris Tibbetts, and Mike Hackett.

Finally, we would like to thank the external reviewers of the book for
their thorough reviews and valuable ideas. They are Frederick C. Berry
(Rose-Hulman Institute of Technology), Mike Bright (Grove City College),
Geoffrey Brooks (Florida State University Panama City Campus) Wils L.
Cooley (West Virginia University), D. J. Godfrey (US Coast Guard
Academy), and Michael Ruane (Boston University).

We hope that you find this book valuable, and that it motivates you to
create great designs. We welcome your comments and input. Please feel
free to email us.

Ralph M. Ford, % \href{mailto:rmf7@psu.edu}{\nolinkurl{rmf7@psu.edu}}

Chris S. Coulston,
% \href{mailto:csc104@psu.edu}{\nolinkurl{csc104@psu.edu}}
