\section{Problems}
\label{section:problems}

\begin{enumerate}
\itemsep0em 
\def\labelenumi{\arabic{enumi}.}
\item
  In your own words, describe the difference between prescriptive and
  descriptive design processes. Cite examples of each.
  
\begin{onlysolution}
\textbf{[R]}
\itshape
Prescriptive design processes “prescribe” an exact sequence of steps and decisions for
realizing a design. There are often decisions that must be made in prescriptive processes for
determining whether to move from one stage to the next, or to move to the next phase.
Descriptive processes describe the general steps needed to achieve a design, but do not
explicitly layout the steps which should be followed to achieve the design.
\end{onlysolution}


  
\item
Describe the relationship between the Problem Identification,
Research, and Requirements Specification phases of the design process.

\begin{onlysolution}
\textbf{[R,A]}
\itshape
Problem Identification, Research, and Requirements Specification are three early phases of
the design process. The overall objective of these phases is to identify a problem, analyze it,
and develop requirements for its solution. The Problem ID phase is where the end-user needs
are determined, while further analysis occurs in the research phase. Both problem and
research phases are used to develop a Requirements Specification that provides the
requirements for those elements that must be satisfied in order for a successful design.
\end{onlysolution}

\item
  Describe the relationship between the Concept Generation and Design
  phases of the design process.
  
\begin{onlysolution}
\textbf{[R,A]}
\itshape
  In Concept Generation, different technical options for solving the design are given – one is
selected to pursue. In the Design Phase, the option selected from Concept Generation is
further clarified and the design architecture is more clearly defined.
\end{onlysolution}
  
\item
  Construct a prescriptive design process for the Problem
  Identification, Research, Specification, Concept, and Design phases of
  the design process. The result should be a flow chart that contains
  decision blocks and iteration as necessary.
  
\begin{onlysolution}
\textbf{[A]}
\itshape
In the prescriptive design process, shown in the figure below, there are two decision points,
one of which occurs after the requirements are determined. The objective in this decision is
to determine whether the requirements satisfy the end-user needs. If not, the needs must be
re-examined and the requirements must be updated as necessary, in order to meet the
customer needs. The other decision occurs after the design is generated. Here, the objective
is to determine if the design satisfies the requirements. If not, a new design concept must be
generated.
\includegraphics[width=1.6in,height=7.3in]{./chapter01/FigSolutions/image19}
\end{onlysolution}
  
  
\item
  Describe the main differences between the VLSI and embedded system
  design processes.
  
\begin{onlysolution}
\textbf{[A]}
\itshape
  VLSI and embedded systems design share similarities and contain
differences. They are both similar in that they have phases for
requirements specifications, system architecture design, and technical
design. The difference between them lies in the technical design, where
the steps depend upon the technology that is being developed. In the
case of VLSI, steps are used to successively refine the design to meet
develop a layout level circuit; however, embedded design requires that
the technical design phase consists of software and hardware co-design.
\end{onlysolution}

\item
Using the library or Internet, conduct research on the spiral software design process.  
\begin{enumerate}
\item Outline the significant elements of the spiral software design process.  
\item Describe the advantages and disadvantages of this relative to the waterfall model? 
\end{enumerate}
Cite all reference used.
  
\begin{onlysolution}
\textbf{[A]}
\itshape
The spiral methodology reflects the relationship of tasks with rapid
prototyping, increased parallelism, and concurrency in design and build
activities$^{[1]}$ The spiral process recognizes that errors will occur in all
stages of the production process and proceeds on this basis$^{[2]}$. It is
agreed that the development processes will have to be revisited multiple
times as the design furthers completion; therefore, unlike the Waterfall
model, this methodology incorporates an iteration cycle, which is
continued until the design is fully complete. A Spiral Development Model
diagram can be found at http://www.hyperthot.com/pm\_sdm.htm as
well at other sites on the Internet.
Embedded in spiral design is the process of refactoring -- changing
software in such a way as to improve structure, but not affect the end
result$^{[3]}$. Overall, the spiral software design model is not as rigid,
concrete, and strict as the Waterfall model; however, this method should
still be planned methodically, with tasks and deliverables identified
for each step within the spiral. The table below lists the advantages
and disadvantages of the spiral design model in reference to the
waterfall model.

\begin{tabular}{ll   ll}\\ 
   & Advantages 						&    & Disadvantages			 \\ \hline
1 & Increased time-to-market 			& 1 & Revisiting the same stages  	\\
2 & Incremental \& Iterative 				& 2 & Requirements are not fully identified  \\
3 & Promotes increase in documentation 		& 3 & Project goal is not initially established  \\
4 & No set structure or phase routine 		&    &   \\
5 & Non-idealistic 					&    &   \\
6 & Not as costly to revisit process steps 		&    &   \\
7 & Primitive to more intricate design 		&    &  \\
8 & 
	\makecell{Allows development to begin \\ w/o full understanding} & &   \\
\end{tabular}

\begin{enumerate}[(1.)]
\item Chapman, James. ``Spiral Methodology.'' \emph{Software Development
Methodology.} 2005. 20 May 2005 http://www.hyperthot.com/pm\_sdm.htm 
\item Culwin, Fintan. ``The Production Process.'' \emph{LAW -- Learn Ada On
the Web}. 1998. 20 May 2005 
http://www.scism.sbu.ac.uk/law/Section1/chap1/s1c1p3.html
\item Hean, Daniel. ``Design through to testing.'' \emph{Content \& Document
Management System.} 2005. 20 May 2005.
http://www.yedit.com/web-content-management-system/400-design-through-to-testing.html
\end{enumerate}
 
 \end{onlysolution}
 

\item
  \textbf{Project Application.} In preparation for project and team
  selection, develop a personal inventory that includes a list of five
  favorite technologies or engineering subjects that you are interested
  in pursuing. Also, list the strengths and weaknesses that you bring to
  a project team.
  
  \begin{onlysolution}
 \textbf{[P]}
\itshape
Note: We find this exercise an important step in starting
students on the path of team formation and project selection and usually
assign it on the first day of class. We setup an electronic bulletin
board for the students and have them post this information publicly for
the whole class to see. Students are then encouraged to review this and
identify potential team-mates. We have also done a variation where each
student is required to determine this information and then make a short
oral presentation (2 minute pitch) to the class, in which they describe
what types of projects they are interested in and what strengths/skills
they can bring to a team.
\end{onlysolution}

\end{enumerate}
