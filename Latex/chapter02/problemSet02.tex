\section{Problems}
\label{section:projectSelectionProblems}


\begin{enumerate}
\def\labelenumi{\arabic{enumi}.}
\item
  In your own words, describe the differences between creative, variant,
  and routine designs.
 \begin{onlysolution}
 \textbf{[R]}
 \itshape
 Creative designs are typically new and innovative design ideas -- those
that did not exist before. Variant designs are variations of existing
designs, with the intent of improving some aspect of the existing
system. Routine designs are concerned with fairly well-known artifacts
for which there is a well-developed design knowledge base.
  \end{onlysolution}
  
\item
  List three guidelines that should be employed when selecting a
  project.
 \begin{onlysolution}
 \textbf{[R]}
 \itshape
 (1) The project must be tied to the mission and vision of the
organization; (2) The project must have payback; (3) The project should
be selected with criterion; (4) The project objectives should be SMART
(Specific, Measurable, Assignable, Realistic, and Time-
Related).
 \end{onlysolution}
  
  
\item
  Assume a customer comes to you with the following
  request---\emph{Design a mechanical arm to pick apples from a tree}.
  What are the assumptions in this statement? Rewrite the request to
  eliminate the assumptions. (This problem was originally posed by
  Edward DeBono {[}Deb70{]}).
\item
  Assume a customer comes to you with the following
  request---\emph{Design an RS-232 networked personal computer
  measurement system to transmit voltage measurements from a remote
  location to a central server.} What are the assumptions this
  statement? Develop a list of questions that you might ask the customer
  to further clarify the problem statement.
\item
  Describe what is meant by a marketing requirement.
\item
  What is the purpose of an objective tree and how is it developed?
\item
  The needs for a garage door opener have been determined to be: safety,
  speed, security, reliability, and noise. Create a pairwise comparison
  to determine the relative weights of the needs. Apply your judgment in
  making the relative comparisons.
\item
  Consider the design of an everyday consumer device such as computer
  printer, digital camera, electric screwdriver, or electric toothbrush.
  Determine the customer needs for the device selected. The deliverables
  should be: 1) marketing requirements, 2) an objective tree, and 3) a
  ranking of the customer needs using pairwise comparison.
\item
  \textbf{Project Application.} Select criteria to be applied for
  selecting a project concept as shown in Example~\ref{example:projectSelectionModel}
  then brainstorm and search to generate project concepts. Rank the top three to five
  concepts against the criteria as presented in Example~\ref{example:projectSelectionModel}.
\item
  \textbf{Project Application.} Determine the needs for the project
  selected. The result should be list of marketing requirements, an
  objective tree, and a ranking of the needs.
\item
  \textbf{Project Application.} Conduct a research survey for your
  project using the guidance presented in Section~\ref{section:needs-identification}. The result should
  be a report summarizing the results of the survey.
\item
  \textbf{Project Application.} Develop a Problem Statement for your
  project concept as outlined in Section~\ref{section:project-application-the-problem-statement}. 
  Apply the processes
  presented in the chapter as appropriate.
\end{enumerate}
