\section{Problems}
\label{section:problems}

\begin{enumerate}
\def\labelenumi{\arabic{enumi}.}

\item
  In your own words, describe what is meant by the work breakdown
  structure.
\item
  Consider the set of activities, duration (in days), and predecessors
  for a project given below.

\begin{table}
\begin{tabular}{|m{2cm}|m{1cm}|m{1cm}|m{1cm}|m{1cm}|m{1cm}|m{1cm}|m{1cm}|m{1cm}|m{1cm}|} \\ \hline
\textbf{Activity} & A & B & C & D & E & F & G & H & I \\ \hline
\textbf{Duration} & 3 & 9 & 6 & 6 & 6 & 3 & 2 & 6 & 7 \\  \hline
\textbf{Predecessors} & - & - & - & A,B & D,B & C & F,E & G & F \\ \hline
\end{tabular}
\end{table}

\begin{enumerate}
\def\labelenumi{\alph{enumi})}
\item
  Develop a network diagram representation for the project.
\item
  Determine the critical path.
\item
  Determine the float time for all activities that are not on the
  critical path.
\end{enumerate}

  \item
    Consider the set of activities, duration (in days), and predecessors
    for a project given below.


\begin{table}
\begin{tabular}{|m{2cm}|m{1cm}|m{1cm}|m{1cm}|m{1cm}|m{1cm}|m{1cm}|m{1cm}|m{1cm}|m{1cm}|m{1cm}|m{1cm}|} \\ \hline
\textbf{Activity} & A & B & C & D & E & F & G & H & I & J & K \\ \hline
\textbf{Duration} & 9 & 12 & 3 & 4 & 5 & 9 & 8 & 3 & 6 & 9 & 1 \\ \hline
\textbf{Predecessors} & - & A & A & B,C & C & B & D & F,D & G & H,I & E \\ \hline
\end{tabular}
\end{table}

\begin{enumerate}
\def\labelenumi{\alph{enumi})}
\item
  Develop a network diagram representation for the project.
\item
  Determine the critical path.
\item
  Determine the float time for all activities that are not on the
  critical path.
\end{enumerate}


  \item
    Explain why a network diagram cannot contain cycles. A cycle is a
    sequence of activities where you can travel back to an activity
    already visited.
  \item
    Describe the advantages and disadvantages of the network diagram and
    Gantt chart representations for a project.
  \item
    Assume that the following data has been determined for the
    development and sale of a new digital thermometer for home use:
    development cost = \$250,000, production investment = \$500,000,
    annual production volume = 20,000 units per year, and the sales
    lifetime is 7 years. Assuming a variable production cost of \$5 per
    unit, determine: (a) the sales price necessary to break even within
    2 years, and (b) the profit expected over the estimated sales
    lifetime.
  \item
    Describe the difference between the cost estimation models in
    equations (7) and (8) and the COCOMO cost estimation model.
  \item
    Consider a software development project that has a team of 50
    software development engineers. The team has proposed a design and
    estimates that it will require 500,000 lines of code to complete the
    project. The average cost to the company for an engineer is \$90,000
    per year, including salary, benefits, and overhead. Estimate (a) the
    time required to complete the project, and (b) the labor costs.
  \item
    Consider a software development project where the team has proposed
    a design and estimates that it will require 200,000 lines of code to
    complete. The average cost to the company for an engineer is
    \$110,000 per year, including salary, benefits, and overhead.
    Estimate (a) the number of engineers needed to complete the project
    within 18 months, and (b) the labor costs.
  \item
    \textbf{Project Application.} Develop a project plan for your
    project. A format and guideline for developing the plan is contained
    in Section 10.7.

\end{enumerate}
