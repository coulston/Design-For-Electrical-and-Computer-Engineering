\section{Problems}
\label{section:reqSpecProblems}
\graphicspath{ {./chapter03/FigSolutions} }

\begin{enumerate}
\def\labelenumi{\arabic{enumi}.}
\item
  Briefly describe the four properties of an engineering requirement.
  
  
\item
  Identify the three levels of standards usage and what is meant by each one.
  
 \begin{onlysolution}
 \textbf{[R]}
 \itshape
  The three levels of standards usage are user, implementation, and
development. The \textbf{user level} simply incorporates the standard
within the design without the need for technical knowledge concerning
the standard. However, the \textbf{implementation level} requires an
in-depth knowledge of the standard -- developing hardware drivers and
ensuring reliability requirements. As with the implementation level, the
\textbf{development level} also requires knowledge of the standard in
order to further develop and modify its predecessor.
\end{onlysolution}
  
  
\item
  For each of the engineering requirements below, determine if it meets
  the properties of abstractness, unambiguous, verifiable, and
  realistic. If a requirement does not satisfy the properties, restate
  it so that it does:


\begin{enumerate}
\def\labelenumi{\alph{enumi})}
\item
  The TV remote control will be easy to use.
  
\begin{onlysolution}
\textbf{[A]}
 \itshape
 Abstractness: \textbf{Yes} -- doesn't give details on implementation\\
Unambiguous: \textbf{Maybe} -- there is not a clear definition of
easy-to-use. It could be possible to develop some metrics for easy to
use, such as size of buttons, number of buttons, etc.\\
Verifiable: \textbf{Maybe -} this relates back to the ambiguity of
easy-to-use. If the easy-to-use property is defined, then it could be
verifiable.
\end{onlysolution}
  
  
\item
  The robot will identify objects in its path using ultrasonic sensors.
  
\begin{onlysolution}
\textbf{[A]}
 \itshape
Abstractness: \textbf{No} -- provides a solution to the problem (ultrasonic sensors)\\
Unambiguous: \textbf{No} -- it will identify objects in its path, is
somewhat clear. However, could be better defined if its path were
defined, as well as the distance of detection\\
Verifiable: \textbf{No} (Because it is not unambiguous.)\\
\textbf{Restatement:} ``The robot will identify objects in its forward
path within 3 feet of the robot.''
\end{onlysolution}

\item
  The car audio amplifier will be encased in aluminum and will operate
  in the automobile environment.
  
  \begin{onlysolution}
  \textbf{[A]}
   \itshape
Abstractness: \textbf{No} -- provides a solution to the problem (aluminum case)\\
Unambiguous: \textbf{No} -- it will operate in an automobile is not
quite clear. Where in the automobile and what size should it be?\\
Verifiable: \textbf{No --} because it is not unambiguous.\\
\textbf{Restatement:} ``The car audio amplifier will operate in the
automobile passgenger compartment and not have a size that exceeds
12''x4''6'' ''
\end{onlysolution}

\item
  The audio amplifier will have a total harmonic distortion that is less
  than 2\%.
  
  \begin{onlysolution}
  \textbf{[A]}
   \itshape
Abstractness: \textbf{Yes} -- doesn't give details on implementation \\
Unambiguous: \textbf{Yes} -- THD \textless{} 2\%  \\
Verifiable: \textbf{Yes}
\end{onlysolution}

\item
  The robot will be able to move at speed of 1 foot/sec in any
  direction.
  
  \begin{onlysolution}
  \textbf{[A]}
   \itshape
Abstractness: \textbf{Yes} -- doesn't give details on implementation \\
Unambiguous: \textbf{No} -- provides two requirements in one statement \\
Verifiable: \textbf{Yes} \\
\textbf{Restatement:} ``The robot will be able to move at a speed of 1 foot/sec.'' or
``The robot will be able to move in any direction.''
\end{onlysolution}

\item
  The system will employ smart power monitoring technology to achieve
  ultra-low power consumption.
  
  \begin{onlysolution}
  \textbf{[A]}
   \itshape
Abstractness: \textbf{No} -- provides a solution to the problem (smart power)\\
Unambiguous: \textbf{Yes} -- it will achieve ultra-low power consumption \\
Verifiable: \textbf{No} -- there is no exact target value on the power \\
\textbf{Restatement:} ``The system will achieve power consumption below XX watts.''
\end{onlysolution}

\item
  The system shall be easy to use by a 12 year old.
  
  \begin{onlysolution}
  \textbf{[A]}
   \itshape
Abstractness: \textbf{Yes} -- doesn't give details on implementation\\
Unambiguous: \textbf{Maybe} -- a 12 year old can use this device is
clear, but as we saw in an earlier problem it is hard to determine ease
of use without some sort of definition.\\
Verifiable: \textbf{Yes}\\
\end{onlysolution}

\item
  The robot must remain operational for 50 years.
  
  \begin{onlysolution}
  \textbf{[A]}
   \itshape
Abstractness: \textbf{Yes} -- doesn't give details on implementation\\
Unambiguous: \textbf{No} -- Failure is a probability-based concept.  A single 
robot always a non-zero chance of failure over an extended periodn of time.\\
Verifiable: \textbf{No} - As a practical matter, your design team would not be
able to perform this test.\\
\end{onlysolution}

\end{enumerate}

  \item
    Provide three example engineering requirements that are technically
    verifiable, but not realistic.
    
  \item
    Describe the difference between \emph{verification} and  \emph{validation}.
    
\begin{onlysolution}
 \textbf{[R]}
 \itshape
Validation is the process of determining if the requirements meet the
needs of the end-user. This answer the question -- are we building the
right product? Verification is the process of measuring or demonstrating
that the requirements are met in the final realization. Verification
answers the question -- are we building the product right (does it meet
the requirements).

Validation is typically harder to determine.
\end{onlysolution}    
    
    
    
    
  \item
    Explain how \emph{validation} is performed for a Requirements Specification.
    
\begin{onlysolution}
  \textbf{[R]}
   \itshape
Validation can be performed by being able to answer the following
questions affirmatively:

\begin{itemize}
\item \textbf{Is each requirement verifiable?} That is can it be measured or
shown in the final system implementation.
\item \textbf{Is each requirement traceable to a user requirement?}
\item \textbf{Is each requirement realistic and technically feasible?} This
may be hard to determine. It can be determined based upon benchmarks or
system prototypes.
\item \textbf{Is the property of orthogonality met for the Requirements
Specification?} Are the requirements established with no redundancy?
\item \textbf{Is the property of completeness met?} Are all the needs of the
end-user addressed in the Requirements Specification?
\item \textbf{Is the property of consistency met?} The Requirements
Specification should not be self-contradictory.
\end{itemize}
\end{onlysolution}    
    
  \item
    Provide an example of a project (real or fictitious) where
    verification is successful, but validation is unsuccessful.
    
%Question 3.8 Use PDF 3.6    
  \item
  \label{list:identifyMarkEngr}
    Consider the design of a common device such as an audio CD player,
    an electric toothbrush, or a laptop computer (or another device that
    you select). Identify potential marketing and engineering
    requirements. Consider those categories presented in 
    Section~\ref{section:engineering-requirements}, as
    well as any others that are applicable to the problem. You do not
    need to select the target values, but should identify the measures
    and units. Present the requirements in a table format as in 
    Table~\ref{table:audioRequireSpec}.
    
 \begin{onlysolution}
   \textbf{[A]}
   \itshape
 \textbf{Marketing Requirements}
\begin{itemize}
\item Should be lightweight
\item Clean teeth well.
\item Have a long battery life.
\item Not shock the user (electric).
\item Be easy to hold
\item Be quiet.
\item Easy to clean.
\item Be lightweight.
\item Allow multiple users.
\end{itemize} 

\begin{tabular}{m{6cm}|m{6cm}}
Engineering Requirements & Notes \\ \hline

%\multicolumn{2}{l}{Performance} \\ \hline

E1.Have \_\_\_ ft-lbs of torque (or translational force, depending upon design). &
This addresses how much force it can apply in cleaning the teeth. This
requirement does require assuming part of the solution.	\\ \hline

E2.Should have a rotational/translational brush speed of \_\_\_
cycles/minute (Note some of the solution assumed here). &
This addresses how quickly it the toothbrush operated.				\\ \hline

E3.Must have a reliability of 95\% at 5 years of service. &
Reliability -- may be a good idea to place an estimate on this. This is
a real guess, and one would have to do more work to determine this one.	\\ \hline

E4. Should emit \textless{} \_\_\_ dB of noise. &
User wanted it to be quiet						 \\ \hline

%\multicolumn{2}{l}{Environmental} \\ \hline

E5. Must work in 100\% humidity (could be submersed). &
Works in a wet environment. Could be submersed.		 \\ \hline

E6.Must be able to withstand \_\_\_ drop from 6 feet and still operate
motor (not brush head). &
User could drop it. 6 feet is typical person height.		\\ \hline

E7.Temperature range of \_\_\_ to \_\_\_\_ degrees Celsius.		\\ \hline

%\multicolumn{2}{l}{Energy} \\ \hline

E8.Should have an operating lifetime of \textgreater{} \_\_\_ hours on a
single battery (or charge). &
How long it will run for.			\\ \hline

%\multicolumn{2}{l}{Packaging/Physical Characteristics} \\ \hline
E9.Toothbrush should weigh less than \_\_ grams. &
Do not want it to be too heavy.			\\ \hline

E10. Should be \_\_\_ cm tall. &
Height should be specified. Should not be too long nor too short.			\\ \hline

%\multicolumn{2}{l}{Cost} \\ \hline

E11. Should cost no more than \$\_\_\_ to produce. &
Cost is virtually always an issue.			\\ \hline
\end{tabular}

 \end{onlysolution}    
    
    
    
    
%Question 3.9 Use PDF 3.7    
  \item
    Develop a marketing-engineering tradeoff matrix for the device
    selected in Problem~\ref{list:identifyMarkEngr}.
    

    
 \begin{onlysolution}
   \textbf{[A]}
   \itshape

\begin{tabular}{l|l|l|l|l|l|l|l|l|l|l|l|l|} 
\multicolumn{2}{l|}{} & 
		 		\rotatebox[origin=c]{90}{E1. Torque} &
  				\rotatebox[origin=c]{90}{E2. Brush Speed}  &
  				\rotatebox[origin=c]{90}{E3. Reliability}  &
  				\rotatebox[origin=c]{90}{E4. Noise}  &
  				\rotatebox[origin=c]{90}{E5. Humidity}  &
  				\rotatebox[origin=c]{90}{E6. Shock Res.}  &
  				\rotatebox[origin=c]{90}{E7. Temp. }  &
  				\rotatebox[origin=c]{90}{E8. Battery Life }  &
  				\rotatebox[origin=c]{90}{E9. Weight }  &
  				\rotatebox[origin=c]{90}{E10. Size }  &
  				\rotatebox[origin=c]{90}{E11. Cost }  \\ \hline
  				
\multicolumn{2}{l|}{}              &  +   & + & +   &  -   & + & +  & + & + & - & -  & -     \\ \hline
M1. Lightweight 		& - &     $\downarrow$ & 	&     $\downarrow$ &    $\downarrow$  & &     $\uparrow$ & & $\downarrow$ &$\uparrow$   & $\uparrow$  &      $\downarrow$ \\ \hline
M2. Cleans Well		& + &     &   $\uparrow$ &     &     & &   & & & $\downarrow$  & $\downarrow$ &    \\ \hline
M3. Long Life		& +&     & 	$\downarrow$ & $\downarrow$     &     & &  $\uparrow$  &$\uparrow$  & $\uparrow$ & $\downarrow$  & &    \\ \hline
M4. Electric shock		& + &     & 	&     &     &  $\uparrow$ &   & & &  & &    \\ \hline
M5. Easy to hold		& + &  $\downarrow$   & 	&     &     & & $\downarrow$  & & $\downarrow$ & $\uparrow$ &$\uparrow$ &  $\downarrow$  \\ \hline
M6. Quiet			& +&  $\downarrow$     &  $\downarrow$ 	&     & $\uparrow$     & &   & & &  &  $\downarrow$  &  $\downarrow$    \\ \hline
M7. Durable		& + &     & 	& $\uparrow$     &     &$\uparrow$  & $\uparrow$   &$\uparrow$  & &  & &  $\downarrow$   \\ \hline
\end{tabular}

 \end{onlysolution}
    
 %Question 3.10 Use PDF 3.8
  \item
    Develop an engineering tradeoff matrix for the device selected in
    Problem~\ref{list:identifyMarkEngr}.
    
 \begin{onlysolution}
    \textbf{[A]}
   \itshape

\begin{tabular}{l|l|l|l|l|l|l|l|l|l|l|l|l|} 
\multicolumn{2}{l|}{} & 
				\rotatebox[origin=c]{90}{E1. Torque} &
  				\rotatebox[origin=c]{90}{E2. Brush Speed}  &
  				\rotatebox[origin=c]{90}{E3. Reliability}  &
  				\rotatebox[origin=c]{90}{E4. Noise Level}  &
  				\rotatebox[origin=c]{90}{E5. Humidity}  &
  				\rotatebox[origin=c]{90}{E6. Phys. Shock }  &
  				\rotatebox[origin=c]{90}{E7. Temp. }  &
  				\rotatebox[origin=c]{90}{E8. Battery Life }  &
  				\rotatebox[origin=c]{90}{E9. Weight }  &
  				\rotatebox[origin=c]{90}{E10. Size }  &
  				\rotatebox[origin=c]{90}{E11. Cost }  \\ \hline
\multicolumn{2}{l|}{}       &  +   & + & +   &  -   & + & +  & + & + & - & -  & -     \\ \hline  				
E1. Torque 		& + &  \cellcolor{lightgray}&  $\downarrow$  & & & & & &  $\downarrow$ & $\downarrow$  & $\downarrow$  &  $\downarrow$  \\ \hline
E2. Brush Speed	& + &\cellcolor{lightgray} & \cellcolor{lightgray}& &$\downarrow$  & & & & $\downarrow$ &  && $\downarrow$\\ \hline
E3.  Reliability	& +&   \cellcolor{lightgray}& \cellcolor{lightgray}& \cellcolor{lightgray}& & $\uparrow$& $\uparrow$& $\uparrow$& & $\downarrow$ &$\downarrow$ &$\downarrow$\\ \hline
E4. Noise Level	& - &  \cellcolor{lightgray}  & \cellcolor{lightgray}& \cellcolor{lightgray}& \cellcolor{lightgray}& & & & & $\downarrow$ & $\downarrow$&$\downarrow$ \\ \hline
E5. Humidity	& + &  \cellcolor{lightgray}& \cellcolor{lightgray}&\cellcolor{lightgray} & \cellcolor{lightgray}& \cellcolor{lightgray}& & & & $\downarrow$ &   & \\ \hline
E6. Phys. Shock	& +&  \cellcolor{lightgray}& \cellcolor{lightgray}&\cellcolor{lightgray} &\cellcolor{lightgray} &\cellcolor{lightgray} & \cellcolor{lightgray}& & & $\downarrow$ & $\downarrow$  & \\ \hline
E7. Temerature	& + & \cellcolor{lightgray}  & \cellcolor{lightgray}&\cellcolor{lightgray} & \cellcolor{lightgray}& \cellcolor{lightgray}& \cellcolor{lightgray}& \cellcolor{lightgray}& \cellcolor{lightgray}& &   & $\downarrow$\\ \hline
E8. Battery Life	& + &  \cellcolor{lightgray} & \cellcolor{lightgray}&\cellcolor{lightgray} &\cellcolor{lightgray} & \cellcolor{lightgray}& \cellcolor{lightgray}&\cellcolor{lightgray} & \cellcolor{lightgray}& $\downarrow$&  $\downarrow$ & $\downarrow$\\ \hline
E9. Weight		& - &   \cellcolor{lightgray} &\cellcolor{lightgray}&\cellcolor{lightgray}& \cellcolor{lightgray}& \cellcolor{lightgray}& \cellcolor{lightgray}& \cellcolor{lightgray}& \cellcolor{lightgray}&  \cellcolor{lightgray}& $\downarrow$  & $\downarrow$\\ \hline
\end{tabular}
 \end{onlysolution}

 %Question 3.11 Use PDF 3.11
  \item
    Develop a list of potential standards that would apply to one of the
    devices proposed in Problem~\ref{list:identifyMarkEngr}, and for each indicate how it would
    apply to the design.
    
\begin{onlysolution}
    \textbf{[A]}
   \itshape
\textbf{Standards for the Electric Toothbrush}
Likely standards for this system include:\\
UL (Underwriters Laboratory) and CE (Common European) safety standards.
This is very common for consumer devices.\\
ADA -- American Dental Association. This would likely be a series of
``standard'' tests before branding ADA approval; therefore, showing that
this system provides sufficient dental treatment.
\end{onlysolution}    

    
  \item
    \textbf{Project Application.} Develop a complete requirements
    document for your project as outlined in 
    Section~\ref{section:project-application-the-requirements-specification}. Make sure that
    the engineering requirements meet the five properties identified in
    the chapter. The team should complete the self-assessment checklist
    in Table~\ref{table:requirementsCheckList}.
    
    
\begin{onlysolution}
    \textbf{[P]}
   \itshape    
   
\textbf{Note:} The \textul{Requirements Specification} is an
important document in the design. Remember that requirement specifications are ``living'' and evolving
documents. Thus it is a good idea to provide design teams with the opportunity to resubmit and
revise the document. We use a two-step submission process. The first submission is
worth 30\% of the specification grade. This is reviewed and resubmitted to the team,
who resubmits, and the second submission constitutes the remaining 70\%. Of course,
if the team gets it right on the first submission, there is no need to resubmit.

\textbf{Constraints}. We have student teams identify at least
five of eight constraints in their specification. They should be
realistic. We don't require that they test each one in the final

realization, but ensure that they are considered (this depends upon the
complexity of the problem). However, the team should be able to that
their system would meet some of the constraints.

Students may also make the counter argument for a constraint. For
example, design team may consider a constraint category, and determine
that it is not applicable. If a clear rationale is given, the team could
document that. If a project has virtually no constraints, then must
question whether or not it is an acceptable project.

\textbf{Standards}. We have students identify standards that
may apply to their project. Of course, they may not know all of
the applicable standards until they get to the design phase.
However, realistic decisions can be made on the standards that will
apply to the project. Some of them may be very beneficial to the design
team. For example, following design standards, such as the IEEE software
design standards can be of great help to the teams.

\textbf{Checklist.} We expect our student teams to also complete
the self-assessment checklist for requirements provided in Table 3.7.

\textbf{Oral Presentations.} After the students complete the
Requirements Specification, we have them make a presentation to a
faculty group. This presentation covers the Problem Statement material
from Chapter 2, the Requirements, Constraints, and Standards. The idea
is for the faculty to make accept/reject the project idea, or more
likely, request changes/corrections.

We also pick one or two teams to present theirs to the entire class
prior to the faculty presentations. This way students can
critique a presentation before hand.

\end{onlysolution}

\end{enumerate}