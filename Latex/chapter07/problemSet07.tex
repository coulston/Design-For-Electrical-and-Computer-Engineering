\section{Problems}
\label{section:testingProblems}

\begin{enumerate}
\def\labelenumi{\arabic{enumi}.}
\item
  Explain the differences between black box and white box testing.

  \begin{onlysolution}
    \textbf{[R]}
    \itshapeThe major difference between these types of testing is knowledge of the internal structure of the circuit. 
    A black box test is performed without knowledge of the system, while white box testing is performed with knowledge 
    of the systems construction. Consequently, black box tests are generally written to check extreme inputs and outputs, 
    and transitional behaviors. White box tests are generally written to force internal nodes of the system to specific values.
  \end{onlysolution}

\item
  Identify a circuit simulator (analog or digital) that you are familiar
  with. Explain the features of this simulator, which increase the
  observability and controllability of the circuit being simulated.

  \begin{onlysolution}
    \textbf{[R]}
    \itshape
    In the VHDL simulator Active HDL a user can explore the design hierarchy during run time and set internal nodes to 
    user defined values. For example, when simulating an arithmetic and logic unit (ALU), the user can supply the top 
    level design unit with data input and control. Then while the ALU is operating on the data, the user can halt the 
    simulator, effectively stopping time. This enhances observability because the user can take their time to observe 
    the state of the system at any point in time. At each level of the design hierarchy, the user can examine the state 
    of any node at that design level; giving the user maximum observability. In addition, the users can override a nodes 
    derived value, giving the user controllability.
  \end{onlysolution}

  \item
  A mobile robot is being built. It uses a two DC motors in a
  differential drive configuration, a microcontroller to control
  movement and an ultrasonic sensor to detect obstacles. The robot is
  built to wander around without bumping into objects. Explain how stubs
  could be used in testing to take the place of incomplete subsystems.
  Be specific.

  \begin{onlysolution}
    \textbf{[R]}
    \itshape
    If the DC motors are being controlled using a PWM schema, then when checking the motor control, an O-scope can take 
    the place of the DC motors. Likewise an O-scope can be used to unit test the ultrasonic range finder. Depending on 
    the complexity of the range finder, a function generator could supply a suitable signal to the MCU in order to test 
    the MCU algorithms.
  \end{onlysolution}

\item
  Consider that you have an op amp integrated circuit package, such as
  the LM741 in Appendix C. What type of testing would be appropriate for
  testing this device? Write a short test plan for doing so.

  \begin{onlysolution}
    \textbf{[R]}
    \itshape
    This question is asking you to test an individual 741 op-amp, not a circuit which might use a LM741. The LM741 is 
    encapsulated inside a ceramic package, consequently its internal nodes are not available for inspection or alternation; 
    it certainly has a low observability. However, its internal organization is given in the technical documentation. 
    Consequently, you could design tests to assert internal nodes of the circuit to particular values and check to see if 
    the output behaves the way that it should for this forced value. For example, the Darlington pair consisting of BJTs Q15 
    and Q17 can be controlled by manipulating a combination of the inverting input (pin 2) and the offset null (pin 5). The 
    state of the Darlington pair can then be checked to see if it influences the output (via BJT Q20) in the expected way. 
    Consequently, you would be writing white-box tests for the circuit.

  \end{onlysolution}

\item
  Explain under what situations a matrix test is appropriate.

  \begin{onlysolution}
    \textbf{[R]}
    \itshape
    A matrix test is appropriate when the tests are structurally similar but differ in their values. For example, imagine 
    testing a digital adder circuit. You might want to check a few hundred different input combinations. Clearly avoiding 
    any unnecessary redundancy in the explanation of the tests would be welcome. This can be achieved with a matrix test.
  \end{onlysolution}
  
\item
  Explain under what situations a step-by-step is test appropriate.

  \begin{onlysolution}
    \textbf{[R]}
    \itshape
    A step-by-step test is appropriate when the set of tests are structurally different. That is a different procedure 
    must be carried out for each test. For example, you might want to test the avoidance behavior of an autonomous robot 
    using different obstacles. Each obstacle might require a unique setup of the robot and the obstacles. In addition, 
    the robot may be designed to respond differently to the various obstacles. In this case it would be wise to give an 
    explicit step-by-step set of instructions for each test.
  \end{onlysolution}

\item
  Consider the stages of unit testing, integration testing, and
  acceptance testing. For each of these stages, identify the
  corresponding requirements that each test should be traceable to.

  \begin{onlysolution}
    \textbf{[R]}
    \itshape
    Each stage of testing is associated with its own portion of the design. This can be seen in the Test Vee shown in 
    Figure 7.1. For example acceptance testing is associated with the requirements specification. Consequently, the 
    acceptance test must trace back to the requirements specification developed in section 3.8. Integration testing is 
    associated with the systems design phase of development. Consequently, the high level design defines the functional 
    and behavioral requirements for the modules. The unit test is associated with the construction of the individual units 
    of the system. The requirements for each module are defined in the low level architectural description of the modules 
    in the functional decomposition.
  \end{onlysolution}

\item
  Consider the case study robot design in Section~\ref{section:case-study-security-robot-design}, 
  which presents an acceptance test for the first system requirement. Develop an
  acceptance test for the second system requirement.

  \begin{onlysolution}
    \textbf{[R]}
    \itshape
    ** Add figure **
  \end{onlysolution}

\item
  Consider the case study robot design in Section~\ref{section:case-study-security-robot-design}. 
  Develop an
  integration test that demonstrates the combined operation of the DC
  motors, MCU, and range finder.

  \begin{onlysolution}
    \textbf{[R]}
    \itshape
    ** Add figure ** 
    Note, that this test is not very formal. You really don’t want to burden your design team with 
    reams of testing especially when the tests may need to be rerun as a result of design changes 
    (e.g. regression testing). This test is to make sure there are no surprises later when the 
    acceptance test for the 1st requirement is run.
  \end{onlysolution}

\item
  Consider the case study robot design in Section~\ref{section:case-study-security-robot-design}. 
  Develop an
  integration test that demonstrates the combined operation of the
  digital compass, MCU, and LCD.

  \begin{onlysolution}
    \textbf{[R]}
    \itshape
    ** Add figure **
  \end{onlysolution}

\item
  Consider the case study robot design in Section~\ref{section:case-study-security-robot-design}. 
  Develop unit tests for range finder, the DC motors, the H-bridges, and the LCD.

  \begin{onlysolution}
    \textbf{[R]}
    \itshape
    ** Add figures **
  \end{onlysolution}

\item
  \textbf{Project Application.} Develop an acceptance test suite for
  your project. The acceptance tests should apply to the engineering
  requirements developed for the system.

  \begin{onlysolution}
    \textbf{[R]}
    \itshape
    Getting students to design good unit tests is difficult. They would much rather start in 
    and build the circuit, considering the time spent up front in designing a test as wasted. 
    However, by requiring students to write a set of unit tests before they start to build the 
    circuits you can eliminate time spent building the wrong part. Its important that the 
    students correctly document the unit tests using the forms shown in Table 7.1 or 7.2. These 
    tests should be written hand-in-hand with the design of the individual units. Since the students 
    have knowledge of the internal organization of the units, these tests should be white-box tests.
  \end{onlysolution}

\item
  \textbf{Project Application.} Develop an integration test suite for
  your project. The integration tests should apply to the higher levels
  of the design architecture and address the interaction between
  functional units.

  \begin{onlysolution}
    \textbf{[R]}
    \itshape
    Integration tests should be written at the same time as the high level architecture is being 
    defined. Consequently, the students are writing black-box tests. These tests should be checking 
    that the interfaces operate correctly with extreme inputs, transitional inputs, and erroneous inputs. 
    We frequently find that groups have to make significant design changes mid-way through the term, long 
    after completing their high level design. Having the students work their top-level design by writing 
    integrations tests helps catch these design snafus earlier in the design cycle.
  \end{onlysolution}

\item
  \textbf{Project Application.} Develop a unit test quite for your
  project. The unit tests should apply to the lowest level units in the
  design.

  \begin{onlysolution}
    \textbf{[R]}
    \itshape
    The acceptance test should be one of the first steps of the design process because they are the requirements 
    documentation. However, developing explicit tests on how the students groups will demonstrate these requirements 
    is an overlooked portion of the design process.
  \end{onlysolution}

\end{enumerate}
