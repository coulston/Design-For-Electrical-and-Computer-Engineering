\chapter{Glossary}
\label{appendix:glossary}

\begin{longtable} { p{3cm} p{11cm}} 

Term & Definition \B \\ \hline
\endhead

\emph{\textbf{acceptance test}} &
An acceptance test verifies that the system meets the  
\emph{\textbf{Requirements Specification}} and stipulates the conditions
under which the customer will accept the system (Chapter~\ref{chapter:testing}). \B  \\

\emph{\textbf{activity on node}} & A form of a \emph{\textbf{network
diagram}} used in a project plan. In the Activity on Node (AON) form,
activities are represented by nodes and the dependencies by arrows
(Chapter~\ref{chapter:projectManagement}). \B  \\

\emph{\textbf{activity}} & An activity is a combination of a
\emph{\textbf{task}} and its associated \emph{\textbf{deliverables}}
that is part of a project plan (Chapter~\ref{chapter:projectManagement}). \B  \\

\emph{\textbf{activity view}} & The activity view is part of the
\emph{\textbf{Unified Modeling Language}}. It is characterized by an
activity diagram; its \emph{\textbf{intention}} is to describe the
sequencing of processes required to complete a task (Chapter~\ref{chapter:behaviorModels}). \B  \\

\emph{\textbf{Analytical Hierarchy Process (AHP)}} & A decision-making
process that combines both quantitative and qualitative inputs. It is
characterized by weighted criteria against which the decision is made, a
numeric ranking of alternatives, and computation of a numerical score
for each alternative (Appendix~\ref{appendix:DecisionMakingAHP} and 
Chapters~\ref{chapter:projectSelection} and \ref{chapter:conceptGen}). \B \\

\emph{\textbf{artifact}} & System, component, or process that is the
end-result of a design (Chapter~\ref{chapter:projectSelection}). \B \\

\emph{\textbf{automated script test}} & An automated script test is a
sequence of commands given to a unit under test. For example, a test may
consist of a sequence of inputs that are provided to the unit, where the
outputs for each input are then verified against pre-specified values
(Chapter~\ref{chapter:testing}). \B \\

\emph{\textbf{baseline requirements}} & The original set of requirements
that are developed for a system (Chapter~\ref{chapter:requirementSpec}). \B \\

\emph{\textbf{black box test}} & A test that is performed without any
knowledge of internal workings of the unit under test (Chapter~\ref{chapter:testing}). \B \\

\emph{\textbf{bottom-up design}} & An approach to system design where
the designer starts with basic components and synthesizes them to
achieve the design objectives. This is contrasted to
\emph{\textbf{top-down}} design (Chapter~\ref{chapter:funcDecomp}). \B \\

\emph{\textbf{Bohrbug}} & Bohrbugs are reliable \emph{\textbf{bugs}}, in
which the error is always in the same place. This is analogous to the
electrons in the Bohr atomic model which assume a definite orbit
(Chapter~\ref{chapter:testing}). \B \\

\emph{\textbf{brainstorming}} & A freeform approach to concept
generation that is often done in groups. This process employs five basic
rules: 1) no criticism of ideas, 2) wild ideas are encouraged, 3)
quantity is stressed over quality, 4) build upon the ideas of others,
and 5) all ideas are recorded (Chapter~\ref{chapter:conceptGen}). \B \\
\emph{\textbf{Brainwriting}} & A variation of
\emph{\textbf{brainstorming}} where a group of people systematically
generate ideas and write them down. Ideas are then passed to other team
members who must build upon them. \B \\
\emph{\textbf{break-even point}} & The break-even point is the point
where the number of units sold is such that there is no profit or loss.
It is determined from the total costs and revenue (Chapter~\ref{chapter:projectManagement}). \B \\
\emph{\textbf{bug}} & A problem or error in a system that causes it to
operate incorrectly (Chapter~\ref{chapter:testing}). \B \\
\emph{\textbf{cardinality ratio}} & The cardinality ratio describes the
multiplicity of the entities in a relationship. It is applied to
\emph{\textbf{entity relationship diagrams}} and Unified Modeling
Language \emph{\textbf{static view diagrams}} (Chapter~\ref{chapter:behaviorModels}). \B \\
\emph{\textbf{class}} & Classes are used in object-oriented system
design. A class defines the attributes and methods (functions) of an
\emph{\textbf{object}} (Chapter~\ref{chapter:behaviorModels}). \B \\
\emph{\textbf{cohesion}} & Refers to how focused a module is---highly
cohesive systems do one or a few things very well. Also see
\emph{\textbf{coupling}} (Chapter~\ref{chapter:funcDecomp}). \B \\
\emph{\textbf{component design specification}} & See
\emph{\textbf{subsystem design specification}} (Chapter~\ref{chapter:requirementSpec}). \B \\
\emph{\textbf{concept fan}} & A graphical tree representation of design
decisions and potential solutions to a problem. Also see
\emph{\textbf{concept table}} (Chapters~\ref{chapter:engineeringDesignProcess} and 
\ref{chapter:conceptGen}). \B \\
\emph{\textbf{concept generation}} & A phase in the \emph{\textbf{design
process}} where many potential solutions to solve the problem are
identified (Chapter~\ref{chapter:engineeringDesignProcess}). \B \\
\emph{\textbf{concept table}} & A tool for generating concepts to solve
a problem. It allows systematic examination of different combinations,
arrangements, and substitutions of different elements for a system. Also
see \emph{\textbf{concept fan}} (Chapter~\ref{chapter:conceptGen}). \B \\
\emph{\textbf{conditional rule-based ethics}} & An ethics system in
which there are certain conditions under which an individual can break a
rule. This is generally because it is believed that the moral good of
the situation outweighs the rule. Also see \emph{\textbf{rule-based
ethics}} (Chapter~\ref{chapter:ethicsLegal}). \B \\
\emph{\textbf{constraint}} & A special type of requirement that
encapsulates a design decision imposed by the environment or a
stakeholder. Constraints often violate the abstractness property of
engineering requirements (Chapter~\ref{chapter:requirementSpec}). \B \\
\emph{\textbf{controllability}} & A principle that applies to testing.
Controllability is the ability to set any node of the system to a
prescribed value (Chapter~\ref{chapter:testing}). \B \\
\emph{\textbf{copyright}} & Copyrights protect published works such as
books, articles, music, and software. A copyright means that others
cannot distribute copyrighted material without permission of the owner
(Chapter~\ref{chapter:ethicsLegal}). \B \\
\emph{\textbf{coupling}} & Modules are coupled if they depend upon each
other in some way to operate properly. Coupling is the extent to which
modules or subsystems are connected. See also \emph{\textbf{cohesion}}
(Chapter~\ref{chapter:funcDecomp}). \B \\
\emph{\textbf{creative design}} & A formal categorization of design
projects. Creative designs represent new and innovative designs 
(Chapter~\ref{chapter:projectSelection}). \B \\
\emph{\textbf{critical path}} & The path with the longest duration in a
project plan. It represents the minimum time required to complete the
project (Chapter~\ref{chapter:projectManagement}). \B \\
\emph{\textbf{cross-functional team}} & Cross-functional teams are those
that are composed of people from different organizational functions,
such as engineering, marketing, and manufacturing. Also see
\emph{\textbf{multi-disciplinary team}} (Chapter~\ref{chapter:teamsTeamwork}). \B \\
\emph{\textbf{data dictionary}} & A dictionary of data contained in a
\emph{\textbf{data flow diagram}}. It contains specific information on
the data flows and is defined using a formal language (Chapter~\ref{chapter:behaviorModels}). \B \\
\emph{\textbf{data flow diagram}} & The \emph{\textbf{intention}} of a
data flow diagram (DFD) is to model the processing and flow of data
inside a system (Chapter~\ref{chapter:behaviorModels}). \B \\
\emph{\textbf{decision matrix}} & A matrix that is used to evaluate and
rank concepts. It integrates both the user-needs and the technical
merits of different concepts (Chapter~\ref{chapter:conceptGen}). \B \\
\emph{\textbf{derating}} & A decrease in the maximum amount of power
that can be dissipated by a device. The amount of derating is based upon
operating conditions, notably increases in temperature (Chapter~\ref{chapter:systemReliability}). \B \\
\emph{\textbf{deliverable}} & Deliverables are entities that are
delivered to the project based upon completion of \emph{\textbf{tasks.}}
Also see \emph{\textbf{activity}} (Chapter~\ref{chapter:projectManagement}). \B \\
\emph{\textbf{descriptive design process}} & Describes typical
activities involved in realizing designs with less emphasis on exact
sequencing than a \textbf{\emph{prescriptive design process} (}
Chapter~\ref{chapter:engineeringDesignProcess}). \B \\
\emph{\textbf{design architecture}} & The main (Level 1) organization
and interconnection of modules in a system (Chapter~\ref{chapter:funcDecomp}). \B \\
\emph{\textbf{design phase}} & Phase in the \emph{\textbf{design
process}} where the technical solution is developed, ultimately
producing a detailed system design. Upon its completion, all major
systems and subsystems are identified and described using an appropriate
model (Chapter~\ref{chapter:engineeringDesignProcess}). \B \\
\emph{\textbf{design process}} & The steps required to take an idea from
concept to realization of the final system. It is a problem-solving
methodology that aims to develop a system that best meets the customer's
need within given constraints (Chapter~\ref{chapter:engineeringDesignProcess}). \B \\
\emph{\textbf{design space}} & The space, or collection, of all possible
solutions to a design problem (Chapter~\ref{chapter:projectSelection}). \B \\
\emph{\textbf{detailed design}} & A phase in the technical design where
the problem can be decomposed no further and the identification of
elements such as circuit components, logic gates, or software code takes
place (Chapter~\ref{chapter:funcDecomp}). \B \\
\emph{\textbf{engineering requirement}} & A requirement of the system
that applies to the technical aspects of the design. An engineering
requirement should be abstract, unambiguous, verifiable, traceable, and
realistic (Chapter~\ref{chapter:requirementSpec}). \B \\
\emph{\textbf{entity relationship diagram (ERD)}} & An ERD is used to
model database systems. The \emph{\textbf{intention}} of an ERD is to
catalog a set of related objects (entities), their attributes, and the
relationships between them (Chapter~\ref{chapter:behaviorModels}). \B \\
\emph{\textbf{entity relationship matrix}} & A matrix that is used to
identify relationships between entities in a database system 
(Chapter~\ref{chapter:behaviorModels}). \B \\
\emph{\textbf{ethics}} & Philosophy that studies
\emph{\textbf{morality}}, the nature of good and bad, and choices to be
made (Chapter~\ref{chapter:ethicsLegal}). \B \\
\emph{\textbf{event}} & An event is an occurrence at a specific time and
place that needs to be remembered and taken into consideration in the
system design (Chapter~\ref{chapter:behaviorModels}). \B \\
\emph{\textbf{event table}} & A table that is used to store information
about \emph{\textbf{events}} in the system. It includes information
regarding the event trigger, the source of the event, and process
triggered by the event (Chapter~\ref{chapter:behaviorModels}). \B \\
\emph{\textbf{failure function}} & The failure function, $F(t)$, is
a mathematical function that provides the probability that a device has
failed at time \emph{t} (Chapter~\ref{chapter:systemReliability}). \B \\
\emph{\textbf{failure rate}} & The failure rate, $\lambda(t)$, for a
device is the expected number of device failures that will occur per
unit time (Chapter~\ref{chapter:systemReliability}). \B \\
\emph{\textbf{fixed costs}} & Fixed costs are those that are constant
regardless of the number of units produced and cannot be directly
charged to a process or activity (Chapter~\ref{chapter:projectManagement}). \B \\
\emph{\textbf{float}} & The amount of \emph{\textbf{slippage}} that an
activity in a project plan can experience without it becoming part of a
new \emph{\textbf{critical path}} (Chapter~\ref{chapter:projectManagement}). \B \\
\emph{\textbf{flowchart}} & A modeling diagram whose intention is to
visually describe a process or algorithm, including its steps and
control (Chapter~\ref{chapter:behaviorModels}). \B \\
\emph{\textbf{functional decomposition}} & A design technique in which a
system is designed by de­termining its overall functionality and then
iteratively decomposing it into component subsys­tems, each with its own
functionality (Chapter~\ref{chapter:funcDecomp}). \B \\
\emph{\textbf{functional requirement}} & A \emph{\textbf{subsystem
design specification}} that describes the inputs, outputs, and
functionality of a system or component (Chapters~\ref{chapter:requirementSpec} and 
\ref{chapter:funcDecomp}). \B \\
\emph{\textbf{Gantt chart}} & Gantt charts are a bar graph
representation of a project plan where the activities are shown on a
timeline (Chapter~\ref{chapter:projectManagement}). \B \\
\emph{\textbf{Heisenbugs}} & Heisenbugs are \emph{\textbf{bugs}} that
are not always reproducible with the same input. This is analogous to
the Heisenberg Uncertainty Principle, in which the position of an
electron is uncertain (Chapter~\ref{chapter:testing}). \B \\
\emph{\textbf{high-performance team}} & A team that significantly
outperforms all similar teams. Part of the Katzenbach and Smith team
model (Chapter~\ref{chapter:teamsTeamwork}). \B \\
\emph{\textbf{integration test}} & An integration test is performed
after the units of a system have been constructed and tested. The
integration test verifies the operation of the integrated system
behavior (Chapter~\ref{chapter:testing}). \B \\
\emph{\textbf{intention}} & The intention of a model is the target
behavior that it aims to describe (Chapter~\ref{chapter:behaviorModels}). \B \\
\emph{\textbf{interaction view}} & The interaction view is part of the
\emph{\textbf{Unified Modeling Language}}. Its \emph{\textbf{intention}}
is to show the interaction between objects. It is characterized by
collaboration and sequence diagrams (Chapter~\ref{chapter:behaviorModels}). \B \\
\emph{\textbf{key attribute}} & An attribute for an entity in a database
system that uniquely identifies an instance of the entity 
(Chapter~\ref{chapter:behaviorModels}). \B \\
\emph{\textbf{lateral thinking}} & A thought process that attempts to
identify creative solutions to a problem. It is not concerned with
developing the solution for the problem, or right or wrong solutions. It
encourages jumping around be­tween ideas. It is contrasted to
\emph{\textbf{vertical thinking}} (Chapter~\ref{chapter:conceptGen}). \B \\
\emph{\textbf{liable}} & Required to pay monetary damages according to
law (Chapter~\ref{chapter:ethicsLegal}). \B \\
\emph{\textbf{marketing requirement (specifications)}} & A statement
that describe the needs of the customer or end-user of a system. They
are typically stated in language that the customer would use (Chapters~\ref{chapter:projectSelection}
and \ref{chapter:requirementSpec}). \B \\
\emph{\textbf{maintenance phase}} & Phase in the \emph{\textbf{design
process}} where the system is maintained, upgraded to add new
functionality, or design problems are corrected (Chapter~\ref{chapter:engineeringDesignProcess}). \B \\
\emph{\textbf{matrix test}} & A matrix test is a test that is suited to
cases where the inputs submitted are structurally the same and differ
only in their values (Chapter~\ref{chapter:testing}). \B \\
\emph{\textbf{mean time to failure}} & The mean time to failure (MTTF)
is a mathematical quantity which answers the question, ``\emph{On
average how long does it take for a device to fail?}'' (Chapter~\ref{chapter:systemReliability}). \B \\
\emph{\textbf{module}} & A block, or subsystem, in a design that
performs a function (Chapter~\ref{chapter:funcDecomp}). \B \\
\emph{\textbf{morals}} & The \emph{\textbf{principles}} of right and
wrong and the decisions that derive from those principles 
(Chapter~\ref{chapter:ethicsLegal}). \B \\
\emph{\textbf{multi-disciplinary team}} & In general, a
multi-disciplinary team is one in which the members have complementary
skills and the team may have representation from multiple technical
disciplines. Also see \emph{\textbf{cross-functional team}} 
(Chapter~\ref{chapter:teamsTeamwork}). \B \\
\emph{\textbf{negligence}} & Failure to exercise caution, which in the
case of design could be in not following reasonable standards and rules
that apply to the situation (Chapter~\ref{chapter:ethicsLegal}). \B \\
\emph{\textbf{network diagram}} & A network diagram is a directed graph
representation of the activities and dependencies between them for a
project (Chapter~\ref{chapter:projectManagement}). \B \\
\emph{\textbf{Nominal Group Technique (NGT)}} & A formal approach to
brainstorming and meeting facilitation. In NGT, each team member
silently generates ideas that are reported out in a round-robin fashion
so that all members have an opportunity to present their ideas. Concepts
are selected by a multi-voting scheme with each member casting a
predetermined number of votes for the ideas. The ideas are then ranked
and discussed (Chapters~\ref{chapter:conceptGen} and \ref{chapter:teamsTeamwork}). \B \\
\emph{\textbf{non-disclosure agreement}} & An agreement that prevents
the signer from disseminating information about a company's products,
services, and trade secrets (Chapter~\ref{chapter:ethicsLegal}). \B \\
\emph{\textbf{object}} & Objects represent both data (attributes) and
the methods (functions) that can act upon data. An object represents a
particular instance of a \emph{\textbf{class}}, which defines the
attributes and methods (Chapter~\ref{chapter:behaviorModels}). \B \\
\emph{\textbf{object type}} & Characteristic of a model used in design.
The object type is capable of encapsulating the actual components used
to construct the system (Chapter~\ref{chapter:behaviorModels}). \B \\
\emph{\textbf{objective tree}} & A hierarchical tree representation of
the customer's needs. The branches of the tree are organized based upon
functional similarity of the needs (Chapter~\ref{chapter:projectSelection}). \B \\
\emph{\textbf{observability}} & This principle applies to testing.
Observability is the ability to observe any node of a system 
(Chapter~\ref{chapter:testing}). \B \\
\emph{\textbf{over-specificity}} & This refers to applying targets for
\emph{\textbf{engineering requirements}} that go beyond what is
necessary for the system. Over-specificity limits the size of the
\emph{\textbf{design space}} (Chapter~\ref{chapter:requirementSpec}). \B \\
\emph{\textbf{pairwise comparison}} & A method of systematically
comparing all customer needs against each other. A comparison matrix is
used for the comparison and the output is a scoring of each of the needs
(Appendix~\ref{appendix:DecisionMakingAHP}, Chapter~\ref{chapter:projectSelection},
 and Chapter~\ref{chapter:conceptGen}). \B \\
\emph{\textbf{parallel system}} & A system that contains multiple
modules performing the same function where a single module would
suffice. The overall system functions correctly when any one of the
submodules is functioning (Chapter~\ref{chapter:systemReliability}). \B \\
\emph{\textbf{patent}} & A patent is a legal device for protecting a
design or invention. If a patent is held for a technology, others cannot
use it without permission of the owner (Chapter~\ref{chapter:ethicsLegal}). \B \\
\emph{\textbf{path-complete coverage}} & Path-complete coverage is where
every possible \emph{\textbf{processing path}} is tested (Chapter~\ref{chapter:testing}). \B \\
\emph{\textbf{performance requirement}} & A particular type of
\emph{\textbf{engineering requirement}} that specifies performance
related measures (Chapter~\ref{chapter:requirementSpec}). \B \\
\emph{\textbf{physical view}} & The physical view is part of the
\emph{\textbf{Unified Modeling Language}}. Its \emph{\textbf{intention}}
is to demonstrate the physical components of a system and how the
logical views map to them. It is characterized by a component and
deployment diagram (Chapter~\ref{chapter:behaviorModels}). \B \\
\emph{\textbf{potential team}} & A team where the sum effort of the team
equals that of the individuals working in isolation. Part of the
Katzenbach and Smith team model (Chapter~\ref{chapter:teamsTeamwork}). \B \\
\emph{\textbf{prescriptive design process}} & An exact process, or
systematic recipe, for realizing a system. Prescriptive design processes
are often algorithmic in nature and expressed using flowcharts with
decision logic (Chapter~\ref{chapter:engineeringDesignProcess}). \B \\
\emph{\textbf{principle}} & Fundamental rules or beliefs that govern
behavior, such as the Golden Rule (Chapter~\ref{chapter:ethicsLegal}). \B \\
\emph{\textbf{problem identification}} & The first phase in the design
process where the problem is identified, the customer needs identified,
and the project feasibility determined (Chapter~\ref{chapter:engineeringDesignProcess}). \B \\
\emph{\textbf{processing path}} & A processing path is a sequence of
consecutive instructions or states encountered while performing a
computation. They are used to develop test cases (Chapter~\ref{chapter:testing}). \B \\
\emph{\textbf{prototyping and construction phase}} & Phase in the
\emph{\textbf{design process}} in which different elements of the system
are constructed and tested. The objective is to model some aspect of the
system, demonstrating functionality to be employed in the final
realization (Chapter~\ref{chapter:engineeringDesignProcess}). \B \\
\emph{\textbf{pseudo-team}} & An under-performing team where the sum
effort of the team is below that of the individuals working in
isolation. Part of the Katzenbach and Smith team model (Chapter~\ref{chapter:teamsTeamwork}). \B \\
\emph{\textbf{Pugh Concept Selection}} & A technique for comparing
design concepts to the user needs. It is an iterative process where
concepts are scored relative to the needs. Each concept is combined,
improved, or removed from consideration in each iteration of the process
(Chapter~\ref{chapter:conceptGen}). \B \\
\emph{\textbf{real team}} & A team where the sum effort of the team
exceeds that of the individuals working in isolation. Part of the
Katzenbach and Smith team model (Chapter~\ref{chapter:teamsTeamwork}). \B \\
\emph{\textbf{redundancy}} & A design has redundancy if it contains
multiple modules performing the same function where a single module
would suffice. Redundancy is used to increase
\emph{\textbf{reliability}} (Chapter~\ref{chapter:systemReliability}). \B \\
\emph{\textbf{reliability}} & Reliability, \emph{R(t)}, is the
probability that a device is functioning properly (has not failed) at
time \emph{t} (Chapter~\ref{chapter:systemReliability}). \B \\
\emph{\textbf{research phase}} & Phase in the \emph{\textbf{design
process}} where research on the basic engineering and scientific
principles, related technologies, and existing solutions for the problem
are explored (Chapter~\ref{chapter:engineeringDesignProcess}). \B \\
\emph{\textbf{Requirements Specification}} & A collection of engineering
and marketing requirements that a system must satisfy in order for it to
meet the needs of the customer or end-user. Alternate terms that are
used for the Requirements Specification are the \emph{Product Design
Specification} and the \emph{Systems Requirements Specification}
(Chapter~\ref{chapter:engineeringDesignProcess} and 3). \B \\
\emph{\textbf{reverse-engineering}} & Process where a device or process
is taken apart to understand how it works (Chapter~\ref{chapter:ethicsLegal}). \B \\
\emph{\textbf{routine design}} & A formal categorization of design
projects. They represent the design of artifacts for which theory and
practice are well-developed (Chapter~\ref{chapter:projectSelection}). \B \\
\emph{\textbf{rule-based ethics}} & Rule-based ethics are based upon a
set of rules that can be applied to make decisions. In the strictest
form, they are considered to be absolute in terms of governing behavior
(Chapter~\ref{chapter:ethicsLegal}). \B \\
\emph{\textbf{satisfice}} & Satisfice means that a solution may meet the
design requirements, but not be the optimal solution (Chapter~\ref{chapter:ethicsLegal}). \B \\
\emph{\textbf{series system}} & A system in which the failure of a
single component (or subsystem) leads to failure of the overall system
(Chapter~\ref{chapter:systemReliability}). \B \\
\emph{\textbf{situational ethics}} & Situational ethics are where
decisions are made based on whether they produce the highest good for
the person (Chapter~\ref{chapter:ethicsLegal}). \B \\
\emph{\textbf{slippage}} & Refers to an activity in a project plan
taking longer than its planned time to complete. See also
\emph{\textbf{critical path}} and \emph{\textbf{float}} (Chapter~\ref{chapter:projectManagement}). \B \\
\emph{\textbf{standards}} & A standard or established way of doing
things. Standards ensure that products work together, from home plumbing
fixtures to the modules in a modern computer. They ensure the health and
safety of products (Chapter~\ref{chapter:requirementSpec}). \B \\
\emph{\textbf{state}} & The state of a system represents the net effect
of all the previous inputs to the system. Since the state characterizes
the history of previous inputs, it is often synonymous with the word
memory (Chapter~\ref{chapter:behaviorModels}). \B \\
\emph{\textbf{state diagram (machine)}} & Diagram used to describe
systems with memory. It consists of states and transitions between
states (Chapter~\ref{chapter:behaviorModels}). \B \\
\emph{\textbf{static view}} & The static view is part of the
\emph{\textbf{Unified Modeling Language}}. The \emph{\textbf{intention}}
of the static view is to show the classes in a system and their
relationships. The static view is characterized by a class diagram
(Chapter~\ref{chapter:behaviorModels}). \B \\
\emph{\textbf{step-by-step test}} & A step-by-step test case is a
prescription for generating a test and checking the results. It is most
effective when the test consists of a complex sequence of steps 
(Chapter~\ref{chapter:testing}). \B \\
\emph{\textbf{strengths and weakness analysis}} & A technique for the
evaluation of potential solutions to a design problem where the
strengths and weaknesses are identified (Chapter~\ref{chapter:conceptGen}). \B \\
\emph{\textbf{structure charts}} & Specialized block diagrams for
visualizing functional software designs. They employ input, output,
transform, coordinate, and composite modules (Chapter~\ref{chapter:funcDecomp}). \B \\
\emph{\textbf{strict liability}} & A form of \emph{\textbf{liability}}
that focuses only on the product itself---if the product contains a
defect that caused harm, the manufacturer is liable (Chapter~\ref{chapter:ethicsLegal}). \B \\
\emph{\textbf{stub}} & A stub is a device that is used to simulate a
subcomponent of a system during testing. Stubs simulate inputs or
monitor outputs from the unit under test (Chapter~\ref{chapter:testing}). \B \\
\emph{\textbf{subsystem design specification}} & Engineering
requirements for subsystems that are constituents of a larger, more
complex system (Chapter~\ref{chapter:requirementSpec}). \B \\
\emph{\textbf{system integration}} & Phase in the \emph{\textbf{design
process}} where all of the subsystems are brought together to produce a
complete working system (Chapter~\ref{chapter:engineeringDesignProcess}). \B \\
\emph{\textbf{task}} & Tasks are actions that accomplish a job as part
of a project plan. Also see \emph{\textbf{activity}} and
\emph{\textbf{deliverable}} (Chapter~\ref{chapter:projectManagement}). \B \\
\emph{\textbf{Team Process Guidelines}} & Guidelines developed by a team
that govern their behavior and identify expectations for performance
(Chapter~\ref{chapter:teamsTeamwork}). \B \\
\emph{\textbf{technical specification}} & A list of the technical
details for a given system, such as operating voltages, processor
architecture, and types of memory. The technical specification is
fundamentally different from a requirement in that it indicates what was
achieved in the end versus what a system needs to achieve from the
outset. (Chapter~\ref{chapter:requirementSpec}). \B \\
\emph{\textbf{test coverage}} & Test coverage is the extent to which the
test cases cover all possible \emph{\textbf{processing paths}} 
(Chapter~\ref{chapter:testing}). \B \\
\emph{\textbf{test phase}} & Phase in the design process where the
system is tested to demonstrate that it meets the requirements (
Chapters~\ref{chapter:engineeringDesignProcess} and \ref{chapter:testing}). \B \\
\emph{\textbf{testable}} & A design is testable when a failure of a
component or subsystem can be quickly located. A testable design is
easier to debug, manufacture, and service in the field (Chapter~\ref{chapter:testing}). \B \\
\emph{\textbf{top-down design}} & An approach to design in which the
designer has an overall vision of what the final system must do, and the
problem is parti­tioned into components, or subsystems that work together
to achieve the overall goal. Then each subsystem is successively refined
and partitioned as necessary. This is contrasted to
\emph{\textbf{bottom-up}} design (Chapter~\ref{chapter:funcDecomp}). \B \\
\emph{\textbf{tort}} & The basis for which a lawsuit is brought forth
(Chapter~\ref{chapter:ethicsLegal}). \B \\
\emph{\textbf{trade secret}} & An approach to protecting intellectual
property where the information is held secretly, without
\emph{\textbf{patent}} protection, so that a competitor cannot access it
(Chapter~\ref{chapter:ethicsLegal}). \B \\
\emph{\textbf{under-specificity}} & This refers to a state of the
\emph{\textbf{Requirements Specification}}. When it is under-specified,
requirements do not meet the needs of the user and/or embody all of the
requirements needed to implement the system (Chapter~\ref{chapter:requirementSpec}). \B \\
\emph{\textbf{Unified Modeling Language (UML)}} & A modeling language
that captures the best practices of object-oriented system design. It
encompasses six different system views that can be used to model
electrical and computer systems (Chapter~\ref{chapter:behaviorModels}). \B \\
\emph{\textbf{unit test}} & A unit test is a test of the functionality
of a system module in isolation. It establishes that a subsystem
performs a single unit of functionality to some specification 
(Chapter~\ref{chapter:testing}). \B \\
\emph{\textbf{use-case view}} & The use-case view is part of the
\emph{\textbf{Unified Modeling Language}}. Its \emph{\textbf{intention}}
is to capture the overall behavior of the system from the user's point
of view and to describe cases in which the system will be used 
(Chapter~\ref{chapter:behaviorModels}). \B \\
\emph{\textbf{utilitarian ethics}} & In utilitarian ethics, decisions
are made based upon the decision that brings about the highest good for
all, relative to all other decisions (Chapter~\ref{chapter:ethicsLegal}). \B \\
\emph{\textbf{validation}} & The process of determining whether the
requirements meet the needs of the user (Chapter~\ref{chapter:requirementSpec}). \B \\
\emph{\textbf{value}} & A value is something that a person or group
believes to be valuable or worthwhile. Also see
\emph{\textbf{principles}} and \emph{\textbf{morals}} (Chapter~\ref{chapter:ethicsLegal}). \B \\
\emph{\textbf{variable costs}} & Variable costs vary depending upon the
process or items being produced, and fluctuate directly with the number
of units produced (Chapter~\ref{chapter:projectManagement}). \B \\
\emph{\textbf{variant design}} & A formal categorization of design
projects. They represent the design of existing systems, where the
intent is to improve performance or add features (Chapter~\ref{chapter:projectSelection}). \B \\
\emph{\textbf{verifiable}} & Refers to a property of an engineering
requirement. It means that there should be a way to measure or
demonstrate that the requirement is met in the final system realization
(Chapter~\ref{chapter:requirementSpec}). \B \\
\emph{\textbf{vertical thinking}} & A linear, or sequential, thought
process that proceeds logically towards the solution of a problem. It
seeks to eliminate incorrect solutions. It is contrasted to
\emph{\textbf{lateral thinking}} (Chapter~\ref{chapter:conceptGen}). \B \\
\emph{\textbf{whistleblower}} & A person who goes outside of their
company or organization to report an ethical or safety problem 
(Chapter~\ref{chapter:ethicsLegal}). \B \\
\emph{\textbf{white box test}} & White box tests are those that are
conducted with knowledge of the internal working of the unit under test
(Chapter~\ref{chapter:testing}). \B \\
\emph{\textbf{work breakdown structure}} & The work breakdown structure
(WBS) is a hierarchical breakdown of the tasks and deliverables that
need to be completed in order to accomplish a project (Chapter~\ref{chapter:projectManagement}). \B \\
\emph{\textbf{working group}} & A group of individuals working in
isolation, who come together occasionally to share information. Part of
the Katzenbach and Smith team model (Chapter~\ref{chapter:teamsTeamwork}). \B \\
\end{longtable}
